\begin{abstract}
Predictive maintenance (PdM) aims to anticipate equipment failures and schedule timely maintenance actions, reducing downtime and increasing operational efficiency.
In high-stakes industries like aerospace and energy, even a small mechanical failure can lead to serious consequences. That’s why predictive maintenance—using data to forecast when machines might break down—has become so important. But most systems today treat it as a simple yes-or-no problem: will a machine fail or not? This oversimplification misses how machines usually degrade—gradually, over time.

Our project takes a more realistic and nuanced approach. Using NASA’s CMAPSS dataset, we cluster raw sensor data to identify five meaningful stages of degradation, from healthy to failure. We then train machine learning models to not only classify which stage a system is currently in but also estimate how long it will remain there. Finally, we combine both outputs into a risk score that helps guide smarter maintenance decisions.

This hybrid approach better mirrors real-world machinery behavior and offers more flexibility than traditional binary models. By recognizing the early signs of wear and quantifying how much time is left before a serious fault, our system empowers operators to take action before things go wrong. It’s a step toward safer, more efficient, and more intelligent maintenance in critical systems.

Predictive maintenance (PdM) for critical systems like turbofan engines often requires understanding gradual degradation beyond simple failure prediction. Traditional methods struggle to capture intermediate health stages. This paper introduces a hybrid PdM framework leveraging the NASA CMAPSS dataset, focusing on creating nuanced, multi-stage degradation labels directly from sensor data using clustering techniques (KMeans/Agglomerative). We propose a four-phase approach: (1) Unsupervised clustering to define five data-driven degradation stages (Normal to Failure). (2) A classification model (e.g., Random Forest, XGBoost) trained on these custom labels to predict the current health stage, addressing class imbalance. (3) A regression model to estimate the remaining useful life (RUL) or time-to-next-stage transition. (4) A novel Risk Score combining stage probability and time-to-failure estimates, normalized for actionable maintenance decisions. Our approach moves beyond binary predictions, offering a more granular view of system health and enabling more informed maintenance scheduling. We demonstrate the effectiveness of this hybrid strategy through experiments on the CMAPSS dataset, evaluating performance using appropriate classification and regression metrics, and visualizing risk trends.

% Keywords (Optional, check conference guidelines): Predictive Maintenance, CMAPSS, Turbofan Engines, Clustering, Multi-Stage Degradation, Hybrid Model, Risk Score, Machine Learning.
\end{abstract}
