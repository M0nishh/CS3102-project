\begin{abstract}
% Abstract should be concise (typically < 200 words for CVPR style)
% Summarize the problem, the proposed approach, key results, and impact.

Predictive maintenance (PdM) for critical systems like turbofan engines often requires understanding gradual degradation beyond simple failure prediction. Traditional methods struggle to capture intermediate health stages. This paper introduces a hybrid PdM framework leveraging the NASA CMAPSS dataset, focusing on creating nuanced, multi-stage degradation labels directly from sensor data using clustering techniques (KMeans/Agglomerative). We propose a four-phase approach: (1) Unsupervised clustering to define five data-driven degradation stages (Normal to Failure). (2) A classification model (e.g., Random Forest, XGBoost) trained on these custom labels to predict the current health stage, addressing class imbalance. (3) A regression model to estimate the remaining useful life (RUL) or time-to-next-stage transition. (4) A novel Risk Score combining stage probability and time-to-failure estimates, normalized for actionable maintenance decisions. Our approach moves beyond binary predictions, offering a more granular view of system health and enabling more informed maintenance scheduling. We demonstrate the effectiveness of this hybrid strategy through experiments on the CMAPSS dataset, evaluating performance using appropriate classification and regression metrics, and visualizing risk trends.

% Keywords (Optional, check conference guidelines): Predictive Maintenance, CMAPSS, Turbofan Engines, Clustering, Multi-Stage Degradation, Hybrid Model, Risk Score, Machine Learning.
\end{abstract}
