\section{Introduction}
\label{sec:intro}
Predictive Maintenance (PdM) is becoming increasingly vital across industries like aerospace, energy, and manufacturing, where unplanned equipment failures can lead to costly downtime, safety hazards, and reduced operational efficiency. The ability to anticipate when a machine is likely to fail—and intervene just in time—offers enormous value. It allows organizations to reduce unnecessary maintenance, extend equipment lifespan, and ensure system reliability, especially in mission-critical environments like aircraft engines or power turbines.

However, most traditional PdM systems are built around a limited view of failure prediction. These systems often reduce the complex, continuous process of machine wear-and-tear into a simple binary decision: whether a failure is imminent or not. While helpful in some cases, this binary classification overlooks the reality that machinery typically degrades gradually over time, passing through various intermediate health states before an actual failure occurs. Ignoring these stages means missing opportunities for earlier and more efficient interventions.

Another common limitation is the dependence on predefined labels—such as Remaining Useful Life (RUL)—which may not truly reflect a system’s operational state. These labels can be rigid and fail to adapt to varied real-world scenarios. As a result, there is a pressing need for more granular, interpretable insights that allow engineers to better understand equipment condition and plan maintenance accordingly. This project addresses these gaps with a more nuanced, data-driven approach to predictive maintenance.

\paragraph{The Challenge of Granular Predictive Maintenance.}
The Challenge of Granular Predictive Maintenance

In industrial environments where reliability is critical—such as aviation, power generation, and manufacturing—predictive maintenance (PdM) plays a vital role in minimizing downtime, reducing costs, and ensuring safety. A single undetected fault in a jet engine, for example, can lead to catastrophic failure with life-threatening consequences. While PdM technologies have advanced significantly, most existing solutions simplify the problem into binary outcomes (failure vs. no failure) or rely solely on Remaining Useful Life (RUL) predictions. These approaches often fail to capture the gradual and complex degradation behavior that real-world machines experience. In practice, machinery doesn't go from "healthy" to "failed" instantly; it deteriorates through multiple, identifiable stages. Overlooking these intermediate states limits the accuracy of predictions and the effectiveness of maintenance decisions.

To address this, our project proposes a hybrid approach using the CMAPSS dataset—a widely recognized benchmark for PdM research. We move beyond binary labels by first applying clustering techniques to raw sensor data to uncover and define multiple degradation stages. Then, we train classification models to detect the current stage of health and regression models to estimate the time until the next degradation phase or failure. Finally, we combine these outputs into a dynamic Risk Score, providing a more nuanced and actionable tool for maintenance planning.

\paragraph{Our Hybrid Approach.}
Our Hybrid Approach

At the heart of our project lies a hybrid predictive maintenance framework designed to mirror the real-world progression of machinery degradation. Instead of relying on predefined or binary failure labels, we introduce a novel, data-driven method to derive custom degradation stages directly from raw sensor data using unsupervised clustering. These stages—ranging from Normal to Failure—form the foundation for two parallel tasks: classification to determine the current health stage of the system, and regression to estimate the time remaining before transitioning to a more severe state. By capturing both what condition the system is in and how long it is likely to stay there, our approach provides a more complete picture of equipment health. We then integrate these insights into a unified Risk Score, which quantifies the urgency of intervention and supports informed maintenance decisions. This combination of unsupervised labeling, multi-output prediction, and actionable risk estimation distinguishes our work as a flexible and scalable solution for modern PdM challenges.

Key Contributions

A novel method to generate multi-stage degradation labels from CMAPSS sensor data via clustering.

A hybrid prediction framework that combines classification and regression to model stage and time-to-failure.

Development of a Risk Score metric that integrates classification probabilities and regression outputs for practical decision support.

A comprehensive evaluation of the framework on the NASA CMAPSS dataset.
\paragraph{Contributions.}
Clearly list the main contributions, aligning with the points above. Use a compact list environment if desired:
\begin{itemize}
	\item A data-driven methodology using unsupervised clustering (KMeans, Agglomerative) to define multi-stage degradation labels for the CMAPSS dataset, independent of standard RUL.
	\item A hybrid predictive framework integrating classification for current health stage assessment and regression for time-to-transition/failure prediction.
	\item The development and evaluation of a Risk Score metric combining probabilistic failure prediction and temporal estimates for enhanced maintenance decision support.
	\item Extensive experimental validation on the CMAPSS dataset demonstrating the feasibility and potential benefits of the proposed approach.
\end{itemize}

% Paper Structure: Outline the rest of the paper.
\paragraph{Paper Organization.}
Section \cref{sec:related_work} reviews relevant literature. Section \cref{sec:methodology} details our proposed methodology. Section \cref{sec:experiments} describes the experimental setup. Section \cref{sec:results} presents and discusses the results. Section \cref{sec:conclusion} concludes the paper.

% --- Optional: Figure Placeholder ---
% \begin{figure}[t]
%   \centering
%   \includegraphics[width=0.8\linewidth]{figures/concept_overview.pdf} % Replace with your concept figure
%   \caption{Conceptual overview of the proposed hybrid predictive maintenance framework.}
%   \label{fig:concept}
% \end{figure}
% ----------------------------------

% ======================================================================
